\documentclass[11pt]{article}

%\usepackage{fullpage}
\usepackage{forest}

\begin{document}

\title{
  ARM Checkpoint \\
  \large \emph{Implementation of an ARM emulator in C}}
\author{Lancelot Blanchard \and Nico D'Cotta \and Kacper Kazaniecki \and William Profit}

\maketitle

\section{Group Organisation}

As a first approach to the task, we decided to meet in the labs and to read through the specs
together in order to make sure every one of us understood correctly what was expected and to share 
our first ideas of implementation. Our first focus was centered on the first task of the project, 
being the implementation of an ARM emulator. We broke down the task into several modules and subproblems 
that could be easier to solve and that would generate an adequate level of abstraction for the implementation of 
the more general problem. These modules are described in the next section. We then distributed the workload 
between the members of the group and started to work.

The group is using \emph{Slack} in order to communicate during the whole project period. We agreed on meeting 
three times a week to coordinate the work and plan the future tasks, while still working on several bits of the 
code separately. We also agreed on a common codestyle (\emph{Microsoft C++ codestyle}) and an important focus of us 
is to make our code as clear as possible.

%HOW WELL WE WORKED - WHAT WE COULD CHANGE (more conventional use of git? ahaha)
Our first deadline was to have completed the first emulator task by Wednesday, May 29th. By this date, our emulator was 
completed and passed 30 tests out of 63. We therefore believe we have worked efficiently, although we could probably have 
debugged the emulator faster in order to pass all the tests on Wednesday. 

\section{Implementation Strategies}

%MODULES
Breaking down the task into subsequent modules was from the start a priority for all of us. We wanted to make sure 
our code was to be easily understandable and reach a high level of abstraction in order to simplify the later task 
of implementing the assembler. After reflection, we ended up chosing the implementing the following structure:  

\bigskip

\hspace*{\fill}
\begin{forest}  
    [\textbf{Emulate}
        [Emulator, for parent={draw, rounded corners}
            [Pipeline, for parent={draw, rounded corners} 
                [1. \emph{fetch}, for parent={draw, rounded corners}]
                [2. Decode, for tree={draw, rounded corners}]
                [3. Execute 
                    [\emph{data\_proc}, for parent={draw, rounded corners}, for tree={draw, rounded corners}]
                    [\emph{mul}]
                    [\emph{data\_trans}]
                    [\emph{branch}]
                    [Shift, for tree={draw, rounded corners}]
                ]
                [Instruction, for tree={draw, rounded corners}]
            ]
            [Loader, for tree={draw, rounded corners}]
        ]
    ]
\end{forest}
\hspace*{\fill}

\bigskip

%WHAT PARTS OF THE EMULATOR WE ARE GOING TO USE FOR THE ASSEMBLER
Our team believes that the next task (implementing an ARM assembler) will very probably use some bits of code that 
we already wrote. As far as we have started the next task, we have used our loader function in order to create a binary file writer and we are fairly certain that a major part of our \emph{decode} module is going to be useful in the conversion from the assembler 
file to the binary file.

%WHAT IS GOING TO BE HARD NEXT? WHAT ARE WE GOING TO APPROACH THESE?
We are fairly confident on the implementation of the following tasks. Our main focus is save as much time as possible for the 
extension, on which we are genuinely eager to work.

\end{document}
