\documentclass[11pt]{article}
\usepackage{hyperref}

%\usepackage{fullpage}

\begin{document}

\title{Final Report\\
  \small{\em{Assembler and Checkm8 chess player}}}
\author{Lancelot Blanchard \and Nico D'Cotta \and Kacper Kazaniecki \and William Profit}

\maketitle

\section{Assembler}

\subsection{Introduction}

In the making of the assembler, we first looked at the emulator to determine
which parts we could reuse. The main part we were able to reuse is the
instruction system. It is an abstract representation of an instruction that is
well adapted for both parts: the emulator reads 32 bits and fills the abstracted
instruction structure and passes it to the executor. On the other hand, the
assembler reads the written instructions and fills the very same structure which
is then easily converted to 32 bits. \\ Applying separation of concerns when
writing the emulator made reusing and adapting the code rather painless.
Although the assembler does resuse the instruction system, the general structure
around it is rather different from the emulator's, we detail it below.

\subsection{Structure}

The program is initially read into a buffer which is then passed to the parser
which will in turn parse the file in two passes. For the first pass, the file
tokenises every line into more abstract symbols such as \em Opcode, Constant,
Address ...\em. We also register labels and constants (values too big to be in
single instructions and which have to be placed at the end of the file) and
place both in respective lists. The second pass reads all the tokenised lines
and substitutes all the labels and constants with the right offsets.\\

Every tokenised line is passed to the encoder which reads the line token by
token and fills the instruction structure with all the data needed for the
perticular instruction type. Once this structure has been filled out, it is very
easy to encode it as 32 bits using bit shifting operations. The 32-bit
instructions are written sequentially to an output buffer which is then written
to the output file. \\ \\
Some of the challenges or bugs we faced included expanding the emulator's memory
to be able to emulate I/O, and writing very similar code to what we did in
emulator but 'backwards' was not always easy.

\section{Extension: Checkm8 chess player}

\subsection{Overview}

We present an automated chess program to play against on a real world board. The
program first uses computer vision to detect the board and extract features to
get the state of the game. We then make use of the AlphaZero algorithm by
DeepMind to make a move. The AlphaZero algorithm runs on the SingularityNET
network which itself runs on the Ethereum blockchain. We thus propose a way for
anyone to defy AlphaZero.

\subsection{Computer Vision}

The first step for the program is to get the state of the board from a camera
and produce an internal representation of it. The goal is therefore to be able
to segment all the cells and determine whether they contain a piece or not.\\ We
initially set out to use line detection to isolate the cells, however there
proved to be a lot of wrong lines detected so we decided to make use of two
markers placed at diagonally opposed corners of the board. With that we can get
the board itself, cropping out the rest of the image. From here, we subdivide
the image into 8 rows and 8 columns to get all the cells.\\ We have tried
several methods to detect whether a cell contains a piece or not. The first one
consists of looking at the standard deviation of the color. We initially look at
the empty board to compute the average standard deviation of an empty cell,
which turns out to be around 1. Then during the game, we compute the standard
deviation of every cell and compare it to the empty standard deviation value and
given a certain threshold, we determine wheter a cell contains a piece or not.
We ran into a couple of problems with this technique, mainly in that if the
cells were not cropped perfectly then, say a white cell would have a bit of
black from the neighbouring cell so the standard deviation would end up passing
the threshold value.\\
Secondly, we tried running a circles detection algorithm on every cell as chess
pieces have a circular shape. However, this proved to be too arbitrary and we
had a lot of false positives as well as false negatives.

\subsection{AlphaZero and SingularityNET}

SingularityNET is a decentralised protocol based on the Ethereum blockchain that
allows to run AI algorithms on the network. Using their AGI token we can query
services of AIs running there and use their answer in our application.It is
believed that AlphaZero has over 3500 elo in chess, compared to about 2800 for
the current human world champion Magnus Carlsen.\\ We therefore make use of the
AlphaZero algorithm developped by Google's DeepMind in 2018. After processing
the image of the board, we determine which move the opponent did and send that
information to the AI which in turn gives us back what move it made.

\subsection{Extension usage and testing}
To compile and run Checkm8, \em OpenCV \em must be installed at
\texttt{/usr/local} see:
\href{https://docs.opencv.org/3.4.6/d7/d9f/tutorial_linux_install.html}{OpenCV install documentation}.
To run Checkm8, % TODO


OpenCV has assertions defined in its own functions that would let us 
know when we did not use them properly, and helped find some tricky bugs.
Fot example, that is how we discovered that we were using the wrong variables
to compute padding, resulting in some negative coordinates for the image
processing.
\texttt{vision.c} had to be debugged through trial and error because we
had to run it with images every time, and vary the images so our code would
work with different lighting conditions, which had a bigger imapct that we 
thought. This made debugging and testing slower thatn we would have liekd.
\subsection{Results}
\section{Conclusion}
\subsection{Reflection on group work}
We mainly communicated through Slack which proved very effective as we were able
to use different channels to discuss different subjects when working on
different parts. This relates to effectively splitting tasks between group
members. We made sure to spend enough time planning before getting into
programming to ensure a good code structure and an optimal distribution of
tasks. We sketched out the entire project as a group so every one would have a
clear idea of the overall structure. Following that, we determined how the
indiviudal components would work together, keeping in mind the idea of making
those modules as black boxes so we can easily assign them to a person and work
independently of each other. For that, we defined the functions that would
bridge the modules together before the actual implementation. This proved to be
very effective and saved a lot of time and debugging.\\
However, we agree in hindsight that we inititially diverged too much for the
extension as we started thinking about making a robotic arm (with Lego mindstorm
pieces). This proved to be very difficult and delayed the project by a lot. The
lesson we got from this is that we should first focus on developping the minimum
viable product before further extending it.\\
Nonetheless, we were able to work efficiently thanks to good task distribution,
organisation and communication.

\subsection{Individual reflections}
\subsubsection{Lancelot Blanchard}
\subsubsection{Nico D'Cotta}
Personally, I believe I have learnt a great deal working on this project, not
only about C but also about good and common coding practices like makefiles and
version control.
It was my first time coding in a group and I think we managed to attain a pretty
pleasant and efficient group dynamic. Not having covered every single part of
the project, I think specialising on certain bits that each of us can do well is
what makes group work work. Small challenges like cross compiling and linking
third party libraries accross operating systems has allowed me to get a glimpse
at what actual software developers have to deal with on a daily basis. \\
I am sure that a lot of what I have learnt will be useful to me in the not too
far future, and am overall satisfied with the work we have produced.


\subsubsection{Kacper Kazaniecki}
\subsubsection{William Profit}
I think we had good group cohesion and cooperation in general. I perticularly
liked structuring the project and splitting tasks, although the project has
taught me I could communicate better in that regard as it is important everyone
understands fully not only what they're working on but also what others are doing. I also
learned a lot about using git in a team through the use of branching and
merging. The project also taught me to use some very useful tools such as
Valgrind and gdb which proved to be instrumental in debugging code and making
sure it's leak-free. Ultimately, I think our group worked very well and everyone
played a good part in the making of the project.

\subsection{Final thoughts}
Overall, we all agree we found this project very eye-opening in how 
groups work in in our field and on tool usage (such as GDB, Make,
Valgrind, Git, etc) that enable us to work towards getting what we 
wanted out of our code. \\
We aimed towards finishing parts I, II and III so we could spend time
working on our extension and be ambitious about it. We regret not being
able to do that to the extend we would have wished, but we found that 
instead we spent the necessary time on the first parts to produce 
efficient and encapsulated quality code. Our extension, although not
as ambitious as we had hoped at first, was a lot of hard work and we 
would like to think is interesting and original enough to stand out.


\end{document}
